%_____________________________________________________________________________________________ 
% LATEX Template: Department of Comp/IT BTech Project Reports
% Sample Chapter
% Sun Mar 27 10:25:35 IST 2011
%
% Note: Itemization, enumeration and other things not shown. A sample figure is included.
%_____________________________________________________________________________________________ 

\chapter{Introduction}
%Th is is a section. We can cite a reference like this: \cite{INTERNET} 	
						% Citation. See references.tex for the entry.
Every academic institution needs timetable for its functioning. Almost all college functions are computerized but timetable generation is still manually done in many institutes. Timetable scheduling is very complex problem involving many constraints. Scheduling timetable manually is time and effort consuming task. The problem is to manage the timetable such that it will satisfy all the constraints. The constraints include clashes of classrooms and timeslots, working hours of particular teacher or classrooms, number of lectures for particular subject, lunch breaks for class, etc.

We have implemented a semi-automated approach for solving this constraint heavy problem for institutions like COEP. It will allow users to make timetable  in a semi-automated manner while ensuring that all the constraints are satisfied.

{\bfseries Limitations of fully automated approach:}
\begin{itemize}
\item All constraints need to be ready at beginning.
\item All data input is mandatory at beginning. 
\item No flexibility.
\item Algorithm works mechanically, cant be modified.
\item Personalization not possible.
\end{itemize}

{\bfseries Limitations of fully manual approach:}
\begin{itemize}
\item All constraints need to be checked manually.
\item Keeping track of data is very difficult.
\item Mechanical tasks need to be done manually.
\item Debugging timetable is really difficult.
\item Takes a lot of time.
\end{itemize}

%\subsection{Vorpal blade}
%And this is a subsection.
			
\section{Problem Statement}

To design and implement an application which can be used to manage the timetable of a department in an educational institute, ensuring that none of the constraints are violated and resources are used optimally while ensuring the flexibility of semi-automated approach.


\section{Requirements}
\begin{itemize}
\item User-friendly cell formatting (colours and styling).
\item Export in \textit{ods, pdf, html} formats.
\item Standard keyboard short-cuts like copy, paste, merge should work.
\item Support merge, unmerge of cells.
\item Ability to manage data of \textit{teachers, venues, classes, subjects} at runtime.
\item Support for batches in class for e.g. \textit{syce-b1, syit-b1} etc.
\item Individual \textit{teacher workload.}
\item Dynamic changing of titles.
\item Show venue utilisation statistics.
\item Check constraints option in menu.
\item \textit{teacher-subject, venue-class and teacher-class} mapping.
\item \textit{teacher, venue, class, subject} data should be imported from file.
\item Improve the grid layout.
\item Define styling for \textit{ods} document.
\item Support typing in cell directly.
\item Open file through command line argument.
\item Show suggestions in pop-up box while typing.
\item Capacity, number of hours, workload should be changeable at runtime.
\item \textit{subject} list should filter as per \textit{class-subject} mapping.
\item Show keyboard short-cuts in menu.
\item Port to web-platform.
\end{itemize}

%\begin{figure}[htbp]			
%\begin{center}
%\input{fig1.latex}			% Be sure to have the input file in the directory
%\caption{A simple figure: Square}	% This will appear in the list of figures
%\label{circle}
%\end{center}
%\end{figure}

%_____________________________________________________________________________________________ 
