%_____________________________________________________________________________________________ 
% LATEX Template: Department of Comp/IT BTech Project Reports
% Sample Chapter
% Sun Mar 27 10:25:35 IST 2011
%
% Note: Itemization, enumeration and other things not shown. A sample figure is included.
%_____________________________________________________________________________________________ 

\chapter{Introduction}
%Th is is a section. We can cite a reference like this: \cite{INTERNET} 	
						% Citation. See references.tex for the entry.
Every academic institution needs timetable for its functioning. Almost all college functions are computerized but timetable generation is still manually done in many institutes. Timetable scheduling is very complex problem involving many constraints. Scheduling timetable manually is time and effort consuming task. The problem is to manage the timetable such that it will satisfy all the constraints. The constraints include clashes of classrooms and timeslots, working hours of particular teacher or classrooms, number of lectures for particular subject, lunch breaks for class, etc.

We have implemented a semi-automated approach for solving this constraint heavy problem for institutions like COEP. It will allow users to make timetable  in a semi-automated manner while ensuring that all the constraints are satisfied.

{\bfseries Limitations of fully automated approach:}
\begin{itemize}
\item All constraints need to be ready at beginning.
\item All data input is mandatory at beginning. 
\item No flexibility.
\item Algorithm works mechanically, can't be modified.
\item Personalization not possible.
\end{itemize}

{\bfseries Limitations of fully manual approach:}
\begin{itemize}
\item All constraints need to be checked manually.
\item Keeping track of data is very difficult.
\item Mechanical tasks need to be done manually.
\item Debugging timetable is really difficult.
\item Takes a lot of time.
\end{itemize}

%\subsection{Vorpal blade}
%And this is a subsection.
			
\section{Problem Statement}

To design and implement an application which can be used to manage the timetable of a department in an educational institute, ensuring that none of the constraints are violated and resources are used optimally while ensuring the flexibility of semi-automated approach.


\section{Requirements}
\begin{itemize}
\item The application should suggest create new or open project suggestion on startup.
\item It should ask for titles (headers) when creating new project, which would be displayed at the top of timetable. 
\item Titles should be changeable dynamically.
\item It should ask for basic constraints like working days per week, number of lectures per day, maximum workload for class and lecture start time for the institution at the time of creation of timetable.
\item There should be three tabs, \textit{Class, Venue and Teacher} which will show classwise, venuewise and teacherwise timetable respectively.
\item Timetable should be displayed in a grid i.e. a table having cells.
\item Cells in a grid should contain abbreviated names of teacher, venue, class and subjects. (and batches if applicable)
\item It should show pop-up box for input in cells, having dropdown style choices for teacher, venue, class and subject.
\item It should support typing in cell directly.
\item There should be user-friendly cell formatting (colours and styling).
\item It should support merge, unmerge of cells.
\item It should have support for batches in class for e.g. \textit{syce-b1, syit-b1} etc.
\item There should be a panel for easy scrolling on left side of screen.
\item Standard keyboard short-cuts like copy, paste, merge should work.
\item Project file should be opened via command line argument.
\item It should have an ability to manage data of \textit{teachers, venues, classes, subjects} at runtime and import it from files.
\item It should have ability to specify individual \textit{teacher workload.}
\item It should support \textit{teacher-subject, venue-class and teacher-class} mapping and importing from file.
\item \textit{Subject} list in the pop-up box should filter as per \textit{class-subject} mapping.
\item While typing in pop-up box, suggestions should be shown.
\item Venue/class capacity, number of hours for a subject, workload for a teacher should be changeable at runtime.
\item Venue utilisation statistics should be shown.
\item There should be a check constraints option in menu.
\item It should export timetable in \textit{ods, pdf, html} formats.
\item It should have ability to define styling for \textit{ods} document.
\item It should show keyboard short-cuts in menu.
\end{itemize}

%\begin{figure}[htbp]			
%\begin{center}
%\input{fig1.latex}			% Be sure to have the input file in the directory
%\caption{A simple figure: Square}	% This will appear in the list of figures
%\label{circle}
%\end{center}
%\end{figure}

%_____________________________________________________________________________________________ 
